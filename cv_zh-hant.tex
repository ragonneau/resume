%%
%% This is cv_zh-hant.tex.
%% It consists of a modified version of the Adaptive CV LaTeX template
%% cv_alessandro_rossini_en.tex of Dr. Alessandro Rossini, available at
%%     https://github.com/alessandrorossini/adcv
%%
%% Copyright 2022 Tom M. Ragonneau.
%% Copyright 2016-2020 Alessandro Rossini.
%%
%% It may be distributed and/or modified under the conditions of the
%% LaTeX Project Public License, either version 1.3 of this license or
%% (at your option) any later version. The latest version of this
%% license is in
%%     http://www.latex-project.org/lppl.txt
%% and version 1.3 or later is part of all distributions of LaTeX
%% version 2005/12/01 or later.
%%
%% This work has the LPPL maintenance status 'maintained'.
%%
%% The Current Maintainer of this work is Tom M. Ragonneau.
%%
\documentclass[a4paper,extended,zh-hant]{adcv}

\addbibresource{ragonneau-bib/strings.bib}
\addbibresource{ragonneau-bib/ragonneau.bib}

\title{Tom M. Ragonneau的簡介}

\adcvname{Tom M.}{Ragonneau}{博士學生}
\adcvtitle{應用數學,香港理工大學}
\adcvaddress{TU834,香港理工大學}{九龍}{香港}{中國}
\adcvwebsite{https://ragonneau.github.io/zh-hant/}{ragonneau.github.io}
\adcvemail{tom.ragonneau}{connect.polyu}{hk}
\adcvphone{(+852) 6995-7651}
\adcvdate{二零二二年一月}

\begin{document}

    \section{學歷}

    \begin{adcvtabletwo}
        \adcvrowtwo{\textbf{博士},香港理工大學,香港,中國}{2019年--現時}
        \ifextended
            \adcvrowmulti{\href{https://www.ugc.edu.hk/eng/ugc/index.html}{大學教育資助委員會}(\href{https://cerg1.ugc.edu.hk/hkpfs/index.html}{HKPFS})贊助。}
            \adcvrowmulti{\href{https://www.zhangzk.net}{張在坤}博士及\href{https://www.polyu.edu.hk/ama/staff/xjchen/ChenXJ.htm}{陳小君}教授監督。}
            \adcvrowmulti{論文主題:Model-based derivative-free optimization methods and software。}
        \fi

        \adcvrowskip

        \adcvrowtwo{\textbf{碩士},Toulouse INP-ENSEEIHT,圖盧茲,法國}{2018年--2019年}
        \ifextended
            \adcvrowmulti{於\href{https://www.enseeiht.fr/fr/formation/masters/performance-in-software-media-and-scientific-computing-psmsc.html}{Performance in Software, Media, and Scientific Computing}畢業。}
        \fi

        \adcvrowskip

        \adcvrowtwo{\textbf{工程碩士},Toulouse INP-ENSEEIHT,圖盧茲,法國}{2016年--2019年}
        \ifextended
            \adcvrowmulti{於高性能計算及大數據畢業。}
        \fi

        \adcvrowskip

        \adcvrowtwo{\textbf{CPGE},Carnot高中,第戎,法國}{2014年--2016年}
        \ifextended
            \adcvrowmulti{為期兩年的深奧學習計劃,以其獲得基本及最關鍵、重要的科學知識。}
        \fi
    \end{adcvtabletwo}

    \section{工作經驗}

    \begin{adcvtabletwo}
        \adcvrowtwo{\textbf{研究助理},香港理工大學,香港,中國}{2019年}
        \ifextended
            \adcvrowmulti{早期開發的\href{https://www.pdfo.net}{PDFO}是由兩種軟件所組成,Python及MATLAB。目的是改良M. J. D. Powell教授研發的無導數優化求解器(包括UOBYQA、NEWUOA、BOBYQA、LINCOA、COBYLA)。}
        \fi

        \adcvrowskip

        \adcvrowtwo{\textbf{工程實習生},Axians Cloud Builder,圖盧茲,法國}{2018年}
        \ifextended
            \adcvrowmulti{透過機器學習工具預測由GPFS管理的計算集群的負載均衡。}
        \fi
    \end{adcvtabletwo}

    \section{出版}

    \ifextended
        % \begin{refsection}
        %     \printbibliography[title={書籍和評論},heading=subbibliography]
        % \end{refsection}

        \begin{refsection}
            \nocite{Benshila_Etal_2020}
            \printbibliography[title={科學期刊},heading=subbibliography]
        \end{refsection}

        % \begin{refsection}
        %     \printbibliography[title={科學專題討論會論文集},heading=subbibliography]
        % \end{refsection}

        % \begin{refsection}
        %     \printbibliography[title={技術專業報告},heading=subbibliography]
        % \end{refsection}

        % \begin{refsection}
        %     \printbibliography[title={報紙及雜誌},heading=subbibliography]
        % \end{refsection}
    \else
        \href{https://scholar.google.com/citations?user=tqxQbU0AAAAJ}{一篇}發在科學期上的論文。
    \fi

    \ifextended
        \section{演說}

        \begin{adcvpresentations}
            \adcvpresentation{PDFO: a cross-platform MATLAB/Python interface for Powell's derivative-free optimization solvers}{SIAM Conference on Optimization(OP21),網上,2021年7月21日}
        \end{adcvpresentations}
    \fi

    \ifextended
        \section{教學}

        \begin{adcvtabletwo}
            \adcvrowtwo{\textbf{數學及科學補習班},香港理工大學,香港,中國}{2020年--2021年}
            \adcvrowmulti{微積分、概率和統計、線性代數。}
            \adcvrowtwo{\textbf{考試監考},香港理工大學,香港,中國}{2019年--現時}
            \adcvrowmulti{監控各種科目的考試和給予評分。}
        \end{adcvtabletwo}
    \fi

    \section{語言}

    \ifextended
        \begin{adcvlanguages}
            \adcvmothertongue{法文}
            \adcvlanguagesheader
            \adcvlanguage{英語}{\adcvCTwo}{\adcvCTwo}{\adcvCOne}{\adcvCOne}{\adcvCOne}
            \adcvlanguage{德文}{\adcvBOne}{\adcvBOne}{\adcvBOne}{\adcvATwo}{\adcvBOne}
            \adcvlanguage{粵語}{\adcvATwo}{\adcvAOne}{\adcvAOne}{\adcvAOne}{\adcvAOne}
            \adcvlanguagesfooter
        \end{adcvlanguages}
    \else
        \textbf{法文}:流利。

        \textbf{英語}:流利。

        \textbf{德文}:略懂。

        \textbf{粵語}:基礎。
    \fi

    \section{個人興趣}

    科學、科技、音樂、電影、登山、潛水。

\end{document}
