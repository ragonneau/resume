%%
%% This is cv_en.tex.
%% It consists of a modified version of the Adaptive CV LaTeX template
%% cv_alessandro_rossini_en.tex of Dr. Alessandro Rossini, available at
%%     https://github.com/alessandrorossini/adcv
%%
%% Copyright 2022 Tom M. Ragonneau.
%% Copyright 2016-2020 Alessandro Rossini.
%%
%% It may be distributed and/or modified under the conditions of the
%% LaTeX Project Public License, either version 1.3 of this license or
%% (at your option) any later version. The latest version of this
%% license is in
%%     http://www.latex-project.org/lppl.txt
%% and version 1.3 or later is part of all distributions of LaTeX
%% version 2005/12/01 or later.
%%
%% This work has the LPPL maintenance status 'maintained'.
%%
%% The Current Maintainer of this work is Tom M. Ragonneau.
%%
\documentclass[a4paper,extended]{adcv}

\addbibresource{ragonneau-bib/strings.bib}
\addbibresource{ragonneau-bib/ragonneau.bib}

\title{Tom M. Ragonneau’s CV}

\adcvname{Tom M.}{Ragonneau}{Ph.D. Candidate}
\adcvtitle{Computational Mathematics, The Hong Kong Polytechnic University}
\adcvaddress{TU834, The Hong Kong Polytechnic University}{Kowloon}{Hong Kong}{China}
\adcvwebsite{https://ragonneau.github.io/en/}{ragonneau.github.io}
\adcvemail{tom.ragonneau}{connect.polyu}{hk}
\adcvphone{(+852) 6995-7651}
\adcvdate{January 2022}

\begin{document}

    \section{Education}

    \begin{adcvtabletwo}
        \adcvrowtwo{\textbf{Ph.D.}, The Hong Kong Polytechnic University, Hong Kong, China}{2019--Present}
        \ifextended
            \adcvrowmulti{Supported by the \href{https://www.ugc.edu.hk/eng/ugc/index.html}{University Grants Committee} of Hong Kong under the \href{https://cerg1.ugc.edu.hk/hkpfs/index.html}{Hong Kong Ph.D. Fellowship Scheme}.}
            \adcvrowmulti{Supervised by Dr. \href{https://www.zhangzk.net}{Zaikun Zhang} and co-supervised by Prof. \href{https://www.polyu.edu.hk/ama/staff/xjchen/ChenXJ.htm}{Xiaojun Chen}.}
            \adcvrowmulti{Subject of the dissertation: Model-based derivative-free optimization methods and software.}
        \fi

        \adcvrowskip

        \adcvrowtwo{\textbf{M.Sc.}, Toulouse INP-ENSEEIHT, Toulouse, France}{2018--2019}
        \ifextended
            \adcvrowmulti{Graduated in \href{https://www.enseeiht.fr/fr/formation/masters/performance-in-software-media-and-scientific-computing-psmsc.html}{Performance in Software, Media, and Scientific Computing}.}
        \fi

        \adcvrowskip

        \adcvrowtwo{\textbf{M.Eng.}, Toulouse INP-ENSEEIHT, Toulouse, France}{2016--2019}
        \ifextended
            \adcvrowmulti{Graduated in High Performance Computing and Big Data.}
        \fi

        \adcvrowskip

        \adcvrowtwo{\textbf{CPGE}, Carnot high-school, Dijon, France}{2014--2016}
        \ifextended
            \adcvrowmulti{Two-year program of profound studies for acquiring elementary but crucial scientific knowledge.}
        \fi
    \end{adcvtabletwo}

    \section{Working experience}

    \begin{adcvtabletwo}
        \adcvrowtwo{\textbf{Research Assistant}, The Hong Kong Polytechnic University, Hong Kong, China}{2019}
        \ifextended
            \adcvrowmulti{Early development of \href{https://www.pdfo.net}{PDFO}, a Python and MATLAB software for using late Prof.\ M. J. D. Powell's derivative-free optimization solvers (including UOBYQA, NEWUOA, BOBYQA, LINCOA, and COBYLA).}
        \fi

        \adcvrowskip

        \adcvrowtwo{\textbf{Trainee Engineer}, Axians Cloud Builder, Toulouse, France}{2018}
        \ifextended
            \adcvrowmulti{Prediction of the load balancing of a computing cluster managed by a GPFS system via machine learning tools.}
        \fi
    \end{adcvtabletwo}

    \section{Publications}

    \ifextended
        % \begin{refsection}
        %     \printbibliography[title={Books and edited proceedings}, heading=subbibliography]
        % \end{refsection}

        \begin{refsection}
            \nocite{Benshila_Etal_2020}
            \printbibliography[title={Peer-reviewed journals},heading=subbibliography]
        \end{refsection}

        % \begin{refsection}
        %     \printbibliography[title={Peer-reviewed conference proceedings},heading=subbibliography]
        % \end{refsection}

        % \begin{refsection}
        %     \printbibliography[title={Project deliverables and technical reports},heading=subbibliography]
        % \end{refsection}

        % \begin{refsection}
        %     \printbibliography[title={Newspapers and magazines},heading=subbibliography]
        % \end{refsection}
    \else
        \href{https://scholar.google.com/citations?user=tqxQbU0AAAAJ}{1 publication} in peer-reviewed conference proceedings and journals.
    \fi

    \ifextended
        \section{Presentations}

        \begin{adcvpresentations}
            \adcvpresentation{PDFO: a cross-platform MATLAB/Python interface for Powell's derivative-free optimization solvers}{SIAM Conference on Optimization (OP21), online, July 21, 2021}
        \end{adcvpresentations}
    \fi

    \ifextended
        \section{Teaching}

        \begin{adcvtabletwo}
            \adcvrowtwo{\textbf{Revision Tutorial Sessions}, The Hong Kong Polytechnic University, Hong Kong, China}{2020--2021}
            \adcvrowmulti{Calculus, Probability \& Statistics, and Linear Algebra.}
            \adcvrowtwo{\textbf{Examination Invigilations}, The Hong Kong Polytechnic University, Hong Kong, China}{2019--Present}
            \adcvrowmulti{Monitored examinations and marked assignments for various subjects.}
        \end{adcvtabletwo}
    \fi

    \section{Languages}

    \ifextended
        \begin{adcvlanguages}
            \adcvmothertongue{French}
            \adcvlanguagesheader
            \adcvlanguage{}{English}{\adcvCTwo}{\adcvCTwo}{\adcvCOne}{\adcvCOne}{\adcvCOne}
            \adcvlanguage{}{German}{\adcvBOne}{\adcvBOne}{\adcvBOne}{\adcvATwo}{\adcvBOne}
            \adcvlanguage{}{Cantonese}{\adcvATwo}{\adcvAOne}{\adcvAOne}{\adcvAOne}{\adcvAOne}
            \adcvlanguagesfooter
        \end{adcvlanguages}
    \else
        \textbf{French}: Native proficiency

        \textbf{English}: Full professional proficiency

        \textbf{German}: Limited working proficiency

        \textbf{Cantonese}: Elementary proficiency
    \fi

    \section{Personal interests}

    Science, technology, music, movies, hiking, scuba diving

\end{document}
