% !TeX program = xetex
% !TeX encoding = utf-8
% !TeX spellcheck = fr-FR
%% cv_fr.tex
%% Copyright 2022 Tom M. Ragonneau
%% Copyright 2016-2020 Alessandro Rossini
%
% This work may be distributed and/or modified under the
% conditions of the LaTeX Project Public License, either version 1.3
% of this license or (at your option) any later version.
% The latest version of this license is in
%   http://www.latex-project.org/lppl.txt
% and version 1.3 or later is part of all distributions of LaTeX
% version 2005/12/01 or later.
%
% This work has the LPPL maintenance status `maintained'.
%
% The Current Maintainer of this work is Tom M. Ragonneau.
\documentclass[a4paper,extended,fr]{adcv}

\addbibresource{ragonneau-bib/strings.bib}
\addbibresource{ragonneau-bib/ragonneau.bib}

\title{CV de Tom M. Ragonneau}

\adcvname{Tom M.}{Ragonneau}{Candidat au Doctorat}
\adcvtitle{Math{\'{e}}matiques Appliqu{\'{e}}es, The Hong Kong Polytechnic University}
\adcvaddress{TU834, The Hong Kong Polytechnic University}{Kowloon}{Hong Kong}{Chine}
\adcvwebsite{https://ragonneau.github.io/fr/}{ragonneau.github.io}
\adcvemail{tom.ragonneau}{connect.polyu}{hk}
\adcvphone{(+852) 6995-7651}
\adcvdate{Janvier 2022}

\begin{document}

    \section{Formation}

    \begin{adcvtabletwo}
        \adcvrowtwo{\textbf{Ph.D.}, The Hong Kong Polytechnic University, Hong Kong, Chine}{2019--Pr{\'{e}}sent}
        \ifextended
            \adcvrowmulti{Financ{\'{e}} par la \href{https://www.ugc.edu.hk/eng/ugc/index.html}{Commission des bourses universitaires} de Hong Kong (\href{https://cerg1.ugc.edu.hk/hkpfs/index.html}{HKPFS}).}
            \adcvrowmulti{Supervis{\'{e}} par Dr. \href{https://www.zhangzk.net}{Zaikun Zhang} et co-supervis{\'{e}} par Prof. \href{https://www.polyu.edu.hk/ama/staff/xjchen/ChenXJ.htm}{Xiaojun Chen}.}
            \adcvrowmulti{Sujet de la dissertation : Model-based derivative-free optimization methods and software.}
        \fi

        \adcvrowskip

        \adcvrowtwo{\textbf{M.Sc.}, Toulouse INP-ENSEEIHT, Toulouse, France}{2018--2019}
        \ifextended
            \adcvrowmulti{Dipl{\^{o}}m{\'{e}} en \href{https://www.enseeiht.fr/fr/formation/masters/performance-in-software-media-and-scientific-computing-psmsc.html}{Performance in Software, Media, and Scientific Computing}.}
        \fi

        \adcvrowskip

        \adcvrowtwo{\textbf{M.Eng.}, Toulouse INP-ENSEEIHT, Toulouse, France}{2016--2019}
        \ifextended
            \adcvrowmulti{Dipl{\^{o}}m{\'{e}} en Calcul de Haute Performance et Big Data.}
        \fi

        \adcvrowskip

        \adcvrowtwo{\textbf{CPGE}, Lyc{\'{e}}e Carnot, Dijon, France}{2014--2016}
        \ifextended
            \adcvrowmulti{Fili{\`{e}}re MPSI/MP, option informatique.}
        \fi
    \end{adcvtabletwo}

    \section{Exp{\'{e}}rience professionnelle}

    \begin{adcvtabletwo}
        \adcvrowtwo{\textbf{Assistant de recherche}, The Hong Kong Polytechnic University, Hong Kong, Chine}{2019}
        \ifextended
            \adcvrowmulti{D{\'{e}}veloppement initial de \href{https://www.pdfo.net}{PDFO}, un logiciel Python et MATLAB pour utiliser les solveurs d'optimisation sans d{\'{e}}riv{\'{e}}e du Prof.\ M. J. D. Powell (comprenant UOBYQA, NEWUOA, BOBYQA, LINCOA, et COBYLA).}
        \fi

        \adcvrowskip

        \adcvrowtwo{\textbf{Ing{\'{e}}nieur stagiaire}, Axians Cloud Builder, Toulouse, France}{2018}
        \ifextended
            \adcvrowmulti{Pr{\'{e}}diction de la r{\'{e}}partition de charge d'un cluster de calcul GPFS via des outils de machine learning.}
        \fi
    \end{adcvtabletwo}

    \section{Publications}

    \ifextended
        % \begin{refsection}
        %     \printbibliography[title={Livres et comptes rendus},heading=subbibliography]
        % \end{refsection}

        \begin{refsection}
            \nocite{Benshila_Etal_2020}
            \printbibliography[title={Revues scientifiques},heading=subbibliography]
        \end{refsection}

        % \begin{refsection}
        %     \printbibliography[title={Actes de colloques scientifiques},heading=subbibliography]
        % \end{refsection}

        % \begin{refsection}
        %     \printbibliography[title={Rapports techniques},heading=subbibliography]
        % \end{refsection}

        % \begin{refsection}
        %     \printbibliography[title={Journaux et magazines},heading=subbibliography]
        % \end{refsection}
    \else
        \href{https://scholar.google.com/citations?user=tqxQbU0AAAAJ}{1 publication} dans une revue scientifique.
    \fi

    \ifextended
        \section{Conf{\'{e}}rences}

        \begin{adcvpresentations}
            \adcvpresentation{PDFO: a cross-platform MATLAB/Python interface for Powell's derivative-free optimization solvers}{SIAM Conference on Optimization (OP21), en ligne, 21 Juillet 2021}
        \end{adcvpresentations}
    \fi

    \ifextended
        \section{Enseignement}

        \begin{adcvtabletwo}
            \adcvrowtwo{\textbf{S{\'{e}}ances de travaux dirigés}, The Hong Kong Polytechnic University, Hong Kong, Chine}{2020--2021}
            \adcvrowmulti{Calcul, Probabilit{\'{e}}s \& Statistiques, et Alg{\`{e}}bre lin{\'{e}}aire.}
            \adcvrowtwo{\textbf{Surveillance d'examens}, The Hong Kong Polytechnic University, Hong Kong, Chine}{2019--Pr{\'{e}}sent}
            \adcvrowmulti{Surveillance d'examens et notations de devoirs pour des sujets vari{\'{e}}s.}
        \end{adcvtabletwo}
    \fi

    \section{Langues}

    \ifextended
        \begin{adcvlanguages}
            \adcvmothertongue{Fran{\c{c}}ais}
            \adcvlanguagesheader
            \adcvlanguage{}{Anglais}{\adcvCTwo}{\adcvCTwo}{\adcvCOne}{\adcvCOne}{\adcvCOne}
            \adcvlanguage{}{Allemand}{\adcvBOne}{\adcvBOne}{\adcvBOne}{\adcvATwo}{\adcvBOne}
            \adcvlanguage{}{Cantonais}{\adcvATwo}{\adcvAOne}{\adcvAOne}{\adcvAOne}{\adcvAOne}
            \adcvlanguagesfooter
        \end{adcvlanguages}
    \else
        \textbf{Fran{\c{c}}ais}: Comp{\'{e}}tence native.

        \textbf{Anglais}: Comp{\'{e}}tence professionnelle compl{\`{e}}te.

        \textbf{Allemand}: Comp{\'{e}}tence professionnelle limit{\'{e}}e.

        \textbf{Cantonais}: Comp{\'{e}}tence {\'{e}}l{\'{e}}mentaire.
    \fi

    \section{Int{\'{e}}r{\^{e}}ts personnels}

    Science, technologie, musique, films, randonn{\'{e}}e, plong{\'{e}}e sous-marine.

\end{document}
